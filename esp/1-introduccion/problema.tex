Para la empresas la búsqueda de financiamiento es vital, dado que de esto, en gran medida, depende su funcionamiento y crecimiento. Sin embargo, los mecanismos actuales para lograrlo son complejos, ya que requieren de condiciones que son prácticamente imposibles de cumplir para empresas pequeñas o nuevas, y son procesos que pueden tomar un tiempo considerable.

En el caso de las startups, uno de estos mecanismos de financiamiento tradicional es a través de créditos, los cuales requieren para ser entregados que dichas empresas tengan cierta cantidad de tiempo de operación, facturación, garantías y otros requisitos que son complejos de conseguir para una empresa pequeña.

Por otro lado, las startups pueden lograr financiamiento mediante la venta privada de parte de sus acciones. En tal proceso, estas empresas se enfrentan a distintas dificultades, que van desde la búsqueda de inversionistas, lo que requiere de tiempo y recursos; la dilusión de la participación de los fundadores; y la necesidad de cambiar la misión y visión de la empresa, lo anterior, con el propósito de adecuarse a incentivos que se alineen con las motivaciones de los inversionistas.

Para las empresas en etapas iniciales, el levantamiento de capital mediante la apertura en bolsa, es prácticamente imposible, ya que se trata de un proceso de altos costos. Existen costos administrativos directos, como el underwriting y cuotas de inscripción, además de costos anuales posteriores, por la elaboración de informes públicos sobre la compañía, servicios de auditoría, cuotas de compra y venta de acciones, etc. Estos costos son, en su mayoría, estándar, tanto para empresas grandes, como pequeñas, lo que implicaría no sólo una dificultad para las organizaciones de mayor tamaño, sino además un gran obstáculo para las organizaciones más pequeñas, que muchas veces no pueden costear tales gastos, haciendo que sea menos probable que estas se listen\cite{pagano}.
