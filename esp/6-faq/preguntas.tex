\textbf{¿Cómo escogen a las empresas que van a tokenizar? }
\vspace{2mm}
\newline
Para postular a GIP deberás inscribir tu empresa en el formulario de GIP, el cual se  encuentra en la siguiente url:  https://gipheadquartes.com. Luego, nuestros encargados se  contactaran para que inicies el proceso de postulación.

\noindent
\vspace{4mm}
\newline
\textbf{¿Cómo puedo comprar CETRONS?}
\vspace{2mm}
\newline
El Initial Exchange Offering se realizará en Orionx, a partir del 23 de diciembre  del 2019. Para esto deberás estar previamente registrado y verificado en el sitio.
¿Cómo funciona el CETRONS y los Tokens ERC-1400 para un inversionista?
Cuando un inversionista desee participar en el financiamiento de los proyectos de GIP, primero tiene que comprar CETRONS en cualquiera de los mercados público que se encuentre. 
Luego, en la plataforma de GIP, habrán distintas ofertas de Tokens ERC-1400, los que tendrán un precio en CETRONS el que se fijará en función de la oferta y demanda que tenga cada Token ERC-1400. 
Cada  vez que un inversionista desee liquidar su posición en Tokens ERC-1400 y obtener a una moneda fiduciaria a cambio, tendrá que vender sus Tokens ERC-1400 que disponga en GIP a cambio de CETRONS. Luego, podrá cambiar  sus CETRONS  a dinero fiduciario en cualquier mercado en el que se encuentre disponible.

\noindent
\vspace{4mm}
\newline
\textbf{¿Cómo funciona el CETRONS y los Tokens ERC-1400 para las empresas?}
\vspace{2mm}
\newline
La empresa deberá dejar en garantía una cantidad de acciones las cuales serán valorizadas según un equipo especializado de GIP. 
Luego, se emitirán tokens ERC-1400 de acuerdo a la cantidad determinada entre GIP y la empresa. Con estos Tokens se hará una oferta inicial en la plataforma de GIP en la cual tendrán un precio en base a CETRONS.
La empresa recibe CETRONS a cambio de sus Tokens ERC-1400, los cuales podrá liquidar en mercados públicos por dinero fiduciario. O también lo podrá usar, para poder obtener otros Tokens ERC-1400 dentro de la plataforma.

\noindent
\vspace{4mm}
\newline
\textbf{¿Por qué los envíos de Token ERC-1400 están restringidos y los CETRONS son libres?}
\vspace{2mm}
\newline
Una de los principales objetivos de GIP, es lograr el correcto desempeño de sus mercados.  Para esto, cada Token ERC-1400 está dotado de una firma que permite que solamente se pueda transar entre billeteras autorizadas por GIP, lo que se logra a través de una firma digital criptográfica. De esta forma se asegura que el intercambio de dicho Token se haga en cumplimiento de regulaciones acorde a la naturaleza de cada Token ERC-1400.

Estas restricciones son innecesarias tratándose de los CETRONS, ya que con ellos, se busca, precisamente abrir la inversión a nuevos actores. Es por ello, que este token puede ser enviado entre distintas billeteras con total libertad. 
