\subsection{Mecanismo de estabilización de precio}
El CETRONS (CTR) no está diseñado para ser una stablecoin, los que por diseño mantienen valores cercanos a \$1 USDEl CETRONS (CTR) no está diseñado para ser una stablecoin, los que por diseño mantienen valores cercanos a \$1 USD. Sin embargo, y dada la función crítica de este token como unidad principal de intercambio, se busca evitar la ocurrencia de fluctuaciones considerables, como ocurre con otros criptoactivos. Para lograr este balance, se ha diseñado un mecanismo de estabilización de precios que funciona de la siguiente forma:

Si el valor promedio del CETRONS (CTR) en todos los mercados donde este se encuentre disponible, aumenta fuertemente su valor., GIP venderá de forma controlada la cantidad necesaria de tokens CETRONS (CTR) hasta que el valor promedio disminuya.

Si el valor promedio del CETRONS (CTR) en todos los mercados donde este se encuentre disponible, disminuye mucho su valor, GIP emitirá la cantidad de tokens ERC-1400 necesarios para incentivar el intercambio de los CETRONS (CTR) por parte de los usuarios, de forma que el valor promedio aumente.
