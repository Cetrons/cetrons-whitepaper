\subsection{CETRONS (CTR) Token}
Con el fin de disminuir los tiempos, costos e intermediarios que tiene el sistema de transacciones internacionales tradicional, se optó por desarrollar una unidad de intercambio propia la cual se denomina “Certificado de transferencia Online”  y moverá todo el ecosistema de GIP. 

Para la creación de esta unidad, se usará tecnología Blockchain, por sus características transaccionales principales: seguridad, inmutabilidad y trazabilidad.

Dicho esto se programará un token llamado CETRONS (CTR). El cual, es un utility token, es decir,  activos digitales con los cuales se financia la red, que se pueden usar para cambiarlos por activos relacionados a esta misma y no otorgan derecho de propiedad \cite{utility}. Asimismo, tiene la característica de ser fungible, que en términos económicos es la propiedad de un bien cuyas unidades individuales son intercambiables e indistinguibles\cite{fungible}.

Se desarrollará en solidity con el estándar ERC-20 el cual está creado para poder desarrollar tokens con las características anteriormente mencionadas\cite{eips20}. Se desplegará en la red de Ethereum, debido a que esta red tiene una trayectoria y un ecosistema ampliamente desarrollado, que permite integraciones con distintos protocolos descentralizados de los cuales CETRONS (CTR) podría beneficiarse en el futuro.

El CETRONS (CTR) se puede enviar entre billeteras de Ethereum con completa libertad, y no está asociado a una regla de envio en particular. Posee mecanismos para su emisión y quema dentro del smart contract, junto con métodos internos que pueden ser mejorados en todo momento por el equipo de desarrollo de GIP.
