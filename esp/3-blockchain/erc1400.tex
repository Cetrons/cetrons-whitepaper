\subsection{Tokens ERC-1400}
Son tokens que comparten características fungibles y no fungibles, son intercambiables, pero no por sus especificaciones individuales\cite{fungible}. Están diseñados para representar intereses de propiedad completa o parcial de un activo o entidad. A diferencia de los tokens ERC-20, estos pueden estar sujetos a distintos mecanismos de control basados en el tipo de activo, categoría e identidad\cite{erc1400}. 

Cada vez que una empresa solicite buscar financiamiento a través de GIP, tendrá que  someterse a una evaluación para determinar si puede participar  de la plataforma. Una vez que pase por este proceso, deberá hacer una prenda sobre parte de sus acciones, las que servirán de respaldo del Token ERC-1400 que se emita especialmente para esa empresa. Posterior a esto GIP generará un espacio de venta para que sus usuarios puedan adquirir el token ERC-1400 de dicha empresa, para que los usuarios de GIP puedan obtenerla por medio de CETRONS.

Los tokens ERC-1400 de GIP, solamente podrán ser operados por instituciones que cumplan con mecanismos de compliance necesarios para poder garantizar la verificación de sus usuarios, a través de los protocolos de KYC y AML necesarios para prevención del uso de activos provenientes de actividades ilícitas.

El mecanismo que se utilizará para poder dar acceso a algunas instituciones será por medio de una firma criptográfica elíptica, la cual se encontrará integrada a los métodos de envío del smart contract. Utilizando la última actualización de Ethereum\cite{eip1108}, los token ERC-1400 correspondiente a cada empresa, poseerán una llave privada con la cual se firmarán las transacciones de cada token ERC-1400 en particular. A partir de esta llave privada, GIP emitirá llaves públicas para cada institución con la que desee transar algún Token en particular. Con esto, se tendrá control de las instituciones financieras que deseen participar del intercambio de los tokens ERC-1400  emitidos por GIP.
