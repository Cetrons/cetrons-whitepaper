\subsection{Cómo opera la plataforma}
Los inversionistas que participen de GIP utilizarán CETRONS (CTR) para intercambiarlos por tokens ERC-1400 dentro de la plataforma.

Las empresas que hayan superado de forma exitosa el proceso de postulación, dispondrán de tokens ERC-1400, los que podrán colocar a la venta a cambio de CETRONS (CTR) dentro de la plataforma.

Las instituciones que deseen financiamiento, constituirán una prenda sobre parte de sus acciones, las que servirán de respaldo del Token ERC-1400 que se emita especialmente para el financiamiento de esa empresa. Los interesados en invertir en ella, deberán ser titulares de CETRONS (CTR), los que, a su vez, deberán utilizar para adquirir los Tokens ERC-1400.

Cada empresa que postule a GIP se le hará envío de una ficha, la cual se tendrá que completar. Posterior a esto, el equipo evaluará si la organización postulante, cumple con los requisitos necesarios para ser listada. Además, se hará un  seguimiento continuo del estado financiero de los proyectos y  se les guiará en el proceso de tokenización. Por otro lado, se desarrollará un  protocolo de integración en el cual las instituciones que tengan mercados activos del token CETRONS (CTR), cumplan con todos los requisitos de compliance necesarios para evitar la recaudación de fondos provenientes de actividades ilícitas.
