\subsection{Interacción de Cetrons, Tokens ERC-1400 y GIP}
En resumen, GIP emitirá un token ERC-20 llamado CETRONS (CTR), el cual lanzará a la venta en diversas plataformas, gracias a su mecanismo de estabilización de precio, se prevé que no tenga grandes variaciones de precio. Este token se puede transar de forma libre, y con el tiempo se expandirá el proyecto para que pueda ser comprado con diferentes divisas.
Por otro lado, las instituciones que busquen financiamiento por medio de GIP y cumplan con todos los requisitos para ser participar de la plataforma, dejarán acciones como prendas para generar tokens ERC-1400 asociados a esta prenda. Luego, por medio de la App de GIP, se abrirá un mercado en la cual  las personas que deseen invertir en dicha empresa usarán sus CETRONS (CTR) para poder adquirir tokens ERC-1400 asociados a la empresa en la que hayan escogido invertir. Los tokens ERC-1400 también se pueden intercambiar en otros mercados, siempre y cuando, las instituciones financieras que operan estos mercados, cumpla con los requisitos necesarios para garantizar la prevención del uso de activos provenientes de actividades ilícitas.
