El sistema de financiamiento tradicional de empresas sufre de ineficiencias que afectan tanto a estas como a sus inversionistas. Esto se traduce en pérdida de dinero y oportunidades para ambas partes.

El presente documento busca explicar cómo la empresa Global Investment Platform, por sus siglas GIP, desarrollará una nueva plataforma de inversión, la cual tiene como objetivo facilitar el acceso de las personas al financiamiento de empresas de innovación, conectándolas con otros usuarios alrededor del mundo que tengan la capacidad de invertir. 

Para esto, se desarrollará una aplicación, la cual funcionará con una unidad de cambio interna llamada “Certificado de transferencia Online”. Esta unidad se representará en un token ERC-20 denominado CETRONS (CTR), el cual se podrá adquirir de forma abierta, desde cualquier exchange que tenga mercados abiertos con este token. Dicho token, al estar desarrollado con tecnología blockchain, tendrá las características de inmutabilidad, transparencia y seguridad sobre los intercambio de esta unidad. 

Por otro lado, a las empresas que deseen buscar financiamiento, y que hayan pasado por un proceso de evaluación dentro de GIP, se les generarán tokens ERC-1400, los cuales podrán ser intercambiados por CETRONS (CTR) dentro de esta misma aplicación.

La meta de GIP es resolver la ineficiencia actual del acceso de las personas a la inversión y el financiamiento de empresas a través de la tecnología, pero con estándares imprescindibles para garantizar las buenas prácticas necesarias para cumplir dicha misión de forma responsable. Dado esto, GIP trabajará al máximo, para poder brindar seguridad tanto a las empresas como a los inversionistas que participen de este nuevo proyecto. 
